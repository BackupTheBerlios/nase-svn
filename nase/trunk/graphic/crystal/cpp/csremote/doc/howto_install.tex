 
\documentclass[german]{article} \usepackage[T1]{fontenc}
\usepackage[latin1]{inputenc} \usepackage{babel}

% z.B. f�r \includegraphics
\usepackage{graphicx}  % Option draft: Boxen statt Bilder


%%%%%%%%%%%%%%%%%%%%%%%%%%%%%%%%%%%%%%%%%%%%%%%%%%%%%%%%%%%%%
\begin{document}

\title{HowTo: Install Crystal Space and CSRemote}

\maketitle


\begin{enumerate}
\item{Aktuellen CVS-Snapshot herunterladen
    (http://crystal.sourceforge.net/cvs-snapshots/)}
\item{Source-Verzeichnis anlegen (.../CS) und hineinwechseln}
\item{Auspacken (tar xzvf cs-current-snapshot.tgz)}
\item{Konfigurieren (in .../CS wechseln, ./configure ausf�hren}
\item{Compilieren (in .../CS wechseln, make -k all)}
\item{Beispielprogramme testen (.../CS/blocks, .../CS/walktest,
    .../CS/simple1), falls diese nicht funktionieren, evtl. mit
    Kommandozeilen-Parametern Konfiguration einstellen (z.B.
    -video=software)}
\item{Installieren (als root in .../CS, make install)}
\item{CS-Umgebungsvariable setzen (/etc/X11/Xsession editieren: export
    CRYSTAL=/usr/local/crystal)}
\item{Dokumentation und Links sind zu finden in
    .../CS/docs/README.html und .../CS/docs/html/cs\_toc.html}
\item {rtfm: 5.1 Developing a Crystal Space Application}
  
\item{Erste selbstcompilierte Beispiel-Applikation ausprobieren
    \begin{enumerate}
    \item {Eigenes Arbeits-Verzeichnis anlegen}
    \item{.../CS/apps/tutorial/simple1/simple1.cpp und
        .../CS/apps/tutorial/simple1/simple1.h dort hineinkopieren}
    \item{rtfm: 5.9.11 Creating an External Crystal Space Application}
    \item{make simple1}
    \item{./simple1}
    \end{enumerate}
  }
  
\item{Erste eigene Beispiel-Anwendung schreiben und csremote einbinden
    \begin{enumerate}
    \item {rtfm: 5.2 Simple Tutorial 1}
    \item{Wie im Tutorial beschrieben die Simple-Anwendung zum Laufen
        bringen}
    \item{in simple.h folgenden Code einf�gen: \\
\begin{verbatim}
...
#include <stdarg.h>
#include "remote.h"
... 
class Simple
{
private:
...
  csRemoteControl* remote; 
  int port; 
...
public:
  Simple (iObjectRegistry* object_reg, int port);
\end{verbatim}
}
    \end{enumerate}
  \item{in simple.cpp einf�gen:\\
\begin{verbatim}
Simple::Simple (iObjectRegistry* object_reg, int port)
{
   Simple::object_reg = object_reg;
   remote = new  csRemoteControl(object_reg);
        Simple::port = port;
}
...
bool Simple::Initialize ()
{
...
  if (!remote->Initialize(g2d, port)) 
    {
      csReport (object_reg, CS_REPORTER_SEVERITY_ERROR,
                "crystalspace.application.simple",
                "Error Initializing Remote Object");
      return false;
    }
  return true;
}
void Simple::SetupFrame ()
{
...
  iCamera* c = view->GetCamera();
  float speed = 0.01;
  csRemoteControlMessage RemoteMessage = CS_REMOTE_NOMESSAGE;
  RemoteMessage = remote->TalkToRemote();

  if (RemoteMessage == CS_REMOTE_KBD_LEFT)
    {
      c->GetTransform ().RotateThis (CS_VEC_ROT_LEFT, speed);
      printf("got a Keybord_left from Remote, speed= %f\n", speed);
    }

  if (RemoteMessage == CS_REMOTE_KBD_RIGHT)
    {
      c->GetTransform ().RotateThis (CS_VEC_ROT_RIGHT, speed);
      printf("got a Keybord_RIGHT, speed=%f\n", speed);
    }

  if (RemoteMessage == CS_REMOTE_SCREENSHOT_REQUEST)
    {
      printf("ScreenShot requested\n");
      remote->SendScreenShot();
    }

  if (kbd->GetKeyState (CSKEY_SPACE))
    {
      remote->ConnectToLocalHost();
    }
...

int main (int argc, char* argv[])
{
  int port=0;
  if (argc > 1) 
  {
      printf("Commandline: %s\n", argv[1]);
      port = atoi(argv[1]);
      printf("PortNummer= %d\n", port);         
  }

  if ((port > 1000) and (port < 9999)) printf("Valid PortNumber\n");
  else port=0;
...
  iObjectRegistry* object_reg =
    csInitializer::CreateEnvironment (argc, argv);
  if (!object_reg) return -1;
\end{verbatim}
{\em ZEILEN-�NDERUNG:}
\begin{verbatim}
  simple = new Simple (object_reg, port);
\end{verbatim}
    } }


\end{enumerate}

\end{document}