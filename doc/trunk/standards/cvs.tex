%
% FILE:     cvs (Master)
%
% LOCATION: /neuro/data/simu/saam/doc/cvs/
%
% AUTHOR:   saam
%
% DATE:     Wed Oct  4 13:49:43 CEST 2000
%
% MODIFICATION HISTORY:
%
%       $Log$
%       Revision 1.1  2000/10/04 12:53:35  saam
%       prelimilary version
%
%
%-
\documentclass[12pt]{article}


% Page Size & Formatting
\usepackage[a4paper,dvips]{geometry}           % page layout [twoside]
\setlength{\parindent}{0pt}                    % dont indent first paragraph
%\renewcommand{\baselinestretch}{2.0}          % double line spacing
%\usepackage{showlabels}                       % shows labels on side margin
\usepackage{showkeys}                          % prints label, ref, cite and bib keys
\usepackage{draftcopy}                         % displays draft in background of page (ps only!)
\usepackage{changebar}                         % changebars at margin via \cbstart ... \cbend (ps only!)


% References & Citations
\usepackage{lastpage}                          % access last page by \pageref{LastPage}
\usepackage[round]{natbib}                     % citestyle: Author (Year) by \cite or Author, Year with \citet
\usepackage{prettyref}                         % generates references like Figure/Table/Equation x, instead of x 
\newrefformat{fig}{Figure~\ref{#1}}
\usepackage{acronym}                           % make sure acronyms are spelled out at least once

% Language
\usepackage[english]{babel}                    % set the language (german, english, ...)
\usepackage[latin1]{inputenc}                  % allow special characters directly in source

% Headers & Footers
\usepackage{fancyheadings}                     % allows individual header and footers
\pagestyle{fancy}
\lhead{\tiny \$Header$ $\$}                    % include path/name/date/RCS/CVS-version in header
\chead{}
\rhead{}
%\lfoot{}
%\cfoot{}
%\rfoot{}


% Fonts
%\usepackage{times}                            % use times, palatino, ...  ps-fonts instead of cm (dvi won't work!!)
%\usepackage{soul}                             % typeset text in spaced out way via \st{...} and more
\usepackage{SIunits}                           % use SI units by their name and produce upright greek symbols

        
% Figures & Captions
\usepackage{graphicx}
\usepackage[footnotesize,nooneline]{caption}
\DeclareGraphicsRule{.tif}{eps}{.bb}{`convert #1 eps:-}
\DeclareGraphicsRule{.gif}{eps}{.bb}{`convert #1 eps:-}
\DeclareGraphicsRule{.bmp}{eps}{.bb}{`convert #1 eps:-}
\DeclareGraphicsRule{.jpg}{eps}{.bb}{`convert #1 eps:-}
% TeX requirements for putting a figure onto a page
% are rather strict (and therefore seldom fulfilled
% the following command make the placement of figures
% more sloppier
\renewcommand{\textfraction}{0.1}
\renewcommand{\topfraction}{0.9}
\renewcommand{\bottomfraction}{0.9}


% Definitions
% the way a bind figures into the document:
\newcommand{\myfig}[5]{\begin{figure}[tbh]\begin{center}\includegraphics[#3]{#2}\caption{\label{#1}{\sc#4.} #5}\end{center}\end{figure}}


% define an acronym by: 
%   \acrodef{CCH}{cross coincidence histogram} 
% following commands produce the shown output:
%   \ac{CCH}  -> cross coincidence histogram (CCH)  % for the first appearance 
%   \ac{CCH}  -> CCH                                % any further appearance
%   \acf{CCH} -> cross coincidence histogram (CCH)  % always
%   \acl{CCH} -> cross coincidence histogram        % always
%   \acs{CCH} -> CCH                                % always
% you can also make a list of acronyms (see documentation)


% integrate figures by:
%   \myfig{label}{filename}{width,height}{short-title}{description}


\begin{document}

\begin{center}
{\Huge\textbf{Concurrent Versions System}}\\[2cm]
Version: \$Id$ $\$ 
\end{center}
This document is a very simple and short introduction to CVS. It
is written for people working with the NASE/MIND system. Please submit
notes, comments and corrections.


\tableofcontents








\section{Other Info}



\section{Installation}

\subsection{Unix}

Set the CVS-Root-Directory to the following location
\begin{verbatim}
bash: export CVSROOT=/vol/neuro/nase/IDLCVS
tcsh: setenv CVSROOT /vol/neuro/nase/IDLCVS
\end{verbatim}

\subsection{Windows}
i wrote it down but can't find it, at the momen





\section{Basic Actions}

\subsection{Create a local working copy}

Change to an arbitrary location in your home directory, where you want to place the copy of the repository. Type 
\begin{verbatim}
cvs checkout nase; cvs checkout mind; cvs checkout mind
\end{verbatim}


\subsection{Update your working copy}
If you have already checked out a working copy, it may that it is outdated, because other users have 
added new routines, new features to existing routines or simply bugfixed some code. To get update to
date just change into a module (\texttt{nase, mind} or \texttt{doc}) and type
\begin{verbatim}
  cvs update -d
\end{verbatim}
The \texttt{-d} flag ensures, that new directories are also create in your working copy. This update
is non-destructive: your local modification will be preserved.      


\subsection{How to modify a file}

The first thing you should do, is to look if other users already work on the file, you are inteding to change.
This is done via
\begin{verbatim}
cvs editors
\end{verbatim}
in the directory, where your file resides. If nobody edits the file, you can tell other to leave their
hands off that file by
\begin{verbatim}
cvs edit filename
\end{verbatim}
Your formerly write-protected version of the file is now writable. If you are finished with
editing, and notice that your changes are nonsense, you can undo your changes and revert to the previous 
version by
\begin{verbatim}
cvs unedit filename
\end{verbatim}
If your changes are all right, and tested you can make the new version visible to all others by
\begin{verbatim}
cvs commit filename
\end{verbatim}


\subsection{Adding new files}

To add a new file to the repository, just create it at the location, where it should
reside. Type
\begin{verbatim}
cvs add filename
\end{verbatim}
To confirm this add, you have to type an additional
\begin{verbatim}
cvs commit filename
\end{verbatim}


\subsection{Deleting the whole working copy}
Sometimes it is useful to delete your local working copy, perhaps because you want to
change its location on your local harddisk. This is done using
\begin{verbatim}
cvs release -d nase
\end{verbatim}
This command ensures, that you do not accidentially erase some of your local modifications. Please
don't just move the working copy, because this confuses CVS.



\section{Working with Tags and Releases}


\subsection{How to create a release}
Functionality and calling conventions of several routines will change
in time. Therefore old simulations may not work together with new versions
of some files. Of you can retrieve the old versions of the files but what
was their revision number? Remember, that every file has its individual
revision number. To restore the complete status of a repository or a sample of files, 
it it possible to create symbolic revision names. This works for single files
\begin{verbatim}
cvs tag super-version-name filename
\end{verbatim}
or for a whole directory:
\begin{verbatim}
cvs tag super-version-name .
\end{verbatim}
The tag names must not contain '\$ , . : ; @'.


\subsection{How to view existing tags}
Existing tags, as well as gerneral informations the as file can be displayed
via
\begin{verbatim}
cvs status -v filename
\end{verbatim}


\subsection{Retrieving a specific release}
\begin{verbatim}
cvs checkout -r release-5-9-97 nase
\end{verbatim}
grabs the complete nase module at the state of 5-9-1997. Another
possibility to checkout certain files is to specifiy the date:
\begin{verbatim}
cvs checkout -D date-string module_or_dir_or_file
\end{verbatim}

The \texttt{date-string} may be

%<P>Als Datum-String sind Sachen wie</P>

%<UL>
%<P>'500000 seconds ago', 'last year', 'yesterday', 'January 23, 1987 10:05pm'</P>
%</UL>

%<P>erlaubt.</P>

%<H2><A NAME="Weitere Infos"></A>Weitere Infos</H2>

%<UL>
%<LI>Nach dem ersten Checkout gibt es eine IDL-Routine names CVS, die eine
%einfache graphische Schnittstelle zur Verf&uuml;gung stellt.</LI>

%<LI><A HREF="http://www.cyclic.com/cyclic-pages/CVS-sheet.html">Allgemein</A></LI>, <A HREF="http://www.loria.fr/~molli/cvs-index.html">Auch Allemein</A> 

%<LI><A HREF="http://papaya.nosc.mil/docs/cvstraining/cvstrain.html">Training
%Manual</A></LI>

%<LI>Manual Pages: <A HREF="http://neuro.physik.uni-marburg.de/cgi-bin/man2html.pp?cgi_command=cvs&cgi_section=1&cgi_keyword=m">cvs(1),
%</A><A HREF="http://neuro.physik.uni-marburg.de/cgi-bin/man2html.pp?cgi_command=cvs&cgi_section=5&cgi_keyword=m">cvs(5)</A></LI>

%<LI>Manual als <A HREF="cvs.ps">Postscript</A></LI>
%</UL>


\end{document}
