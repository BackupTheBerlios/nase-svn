%
% FILE:     standards (Master)
%
% LOCATION: /neuro/cartman/user/saam/nase/doc/standards/
%
% AUTHOR:   saam
%
% DATE:     Thu Aug 24 10:29:40 CEST 2000
%
% MODIFICATION HISTORY:
%
%       $Log$
%       Revision 1.3  2000/08/25 10:09:22  saam
%            compatability added
%
%       Revision 1.2  2000/08/24 16:46:12  saam
%             extended the messaging section
%
%       Revision 1.1  2000/08/24 10:01:32  saam
%             first version
%
%
%-
\documentclass[12pt]{article}


% Page Size & Formatting
\usepackage[a4paper,dvips]{geometry}           % page layout [twoside]
\setlength{\parindent}{0pt}                    % dont indent first paragraph
%\renewcommand{\baselinestretch}{2.0}          % double line spacing
%\usepackage{showlabels}                       % shows labels on side margin
\usepackage{showkeys}                          % prints label, ref, cite and bib keys
\usepackage{draftcopy}                         % displays draft in background of page (ps only!)
\usepackage{changebar}                         % changebars at margin via \cbstart ... \cbend (ps only!)


% References & Citations
\usepackage{lastpage}                          % access last page by \pageref{LastPage}
\usepackage[round]{natbib}                     % citestyle: Author (Year) by \cite or Author, Year with \citet
\usepackage{prettyref}                         % generates references like Figure/Table/Equation x, instead of x 
\newrefformat{fig}{Figure~\ref{#1}}
\usepackage{acronym}                           % make sure acronyms are spelled out at least once

% Language
\usepackage[english]{babel}                    % set the language (german, english, ...)
\usepackage[latin1]{inputenc}                  % allow special characters directly in source

% Headers & Footers
\usepackage{fancyheadings}                     % allows individual header and footers
\pagestyle{fancy}
\lhead{\tiny \$Header$ $\$}                    % include path/name/date/RCS/CVS-version in header
\chead{}
\rhead{}
%\lfoot{}
%\cfoot{}
%\rfoot{}


% Fonts
%\usepackage{times}                            % use times, palatino, ...  ps-fonts instead of cm (dvi won't work!!)
%\usepackage{soul}                             % typeset text in spaced out way via \st{...} and more
\usepackage{SIunits}                           % use SI units by their name and produce upright greek symbols

        
% Figures & Captions
\usepackage{graphicx}
\usepackage[footnotesize,nooneline]{caption}
\DeclareGraphicsRule{.tif}{eps}{.bb}{`convert #1 eps:-}
\DeclareGraphicsRule{.gif}{eps}{.bb}{`convert #1 eps:-}
\DeclareGraphicsRule{.bmp}{eps}{.bb}{`convert #1 eps:-}
\DeclareGraphicsRule{.jpg}{eps}{.bb}{`convert #1 eps:-}
% TeX requirements for putting a figure onto a page
% are rather strict (and therefore seldom fulfilled
% the following command make the placement of figures
% more sloppier
\renewcommand{\textfraction}{0.1}
\renewcommand{\topfraction}{0.9}
\renewcommand{\bottomfraction}{0.9}


% Definitions
% the way a bind figures into the document:
\newcommand{\myfig}[5]{\begin{figure}[tbh]\begin{center}\includegraphics[#3]{#2}\caption{\label{#1}{\sc#4.} #5}\end{center}\end{figure}}


% define an acronym by: 
%   \acrodef{CCH}{cross coincidence histogram} 
% following commands produce the shown output:
%   \ac{CCH}  -> cross coincidence histogram (CCH)  % for the first appearance 
%   \ac{CCH}  -> CCH                                % any further appearance
%   \acf{CCH} -> cross coincidence histogram (CCH)  % always
%   \acl{CCH} -> cross coincidence histogram        % always
%   \acs{CCH} -> CCH                                % always
% you can also make a list of acronyms (see documentation)


% integrate figures by:
%   \myfig{label}{filename}{width,height}{short-title}{description}


\begin{document}

{\Huge\textbf{N.A.S.E. / M.I.N.D. Standards}}

%%%%%%%%%%%%%%%%%%%%%%%
%%%%%%%%%%%%%%%%%%%%%%%
\section{Documentation}
%%%%%%%%%%%%%%%%%%%%%%%
%%%%%%%%%%%%%%%%%%%%%%%

%%%%%%%%%%%%%%%%%%
%%%%%%%%%%%%%%%%%%
\section{Routines}
%%%%%%%%%%%%%%%%%%
%%%%%%%%%%%%%%%%%%
\subsection{Naming}
\subsection{Compatibility between IDL versions}
Different types of compatilibity problems are known between IDL versions. First (the harmless one), higher version include routines, which formerly had to be implemented by hand. Furthermore, already existing routines are extended in their functionality, making several hand-written wrappers obsolete. Last, but most problematic are undocumented (silent) changes in the behaviour of routines, i.e. color management, handling of the \texttt{\_EXTRA} keyword, passing of undefined arguments. The following guidelines are a compromise between functionality and compatability:
\begin{itemize}
\item If a new IDL routines makes a NASE routine obsolete, disable the NASE routine, generate a message about the fact, remove the routine after 2 months from the repository.
\item For changes in IDL's behaviour: all routines must work with (and only with) the current IDL version. The only exception is the NASE/MIND simulation kernel.
\end{itemize}

\subsection{Error Handling and Messaging}
\begin{itemize}
%
\item Wrong number, type, dimensions of positional or keyword parameters should stop the routine at the position of the calling program segment. For this you should place 
\begin{verbatim}
On_Error, 2
\end{verbatim}
(after testing and debugging, of course) at the very beginning of your routine. To produce the actual error message, use
\begin{verbatim}
Console, 'error message', /FATAL .
\end{verbatim}
%
\item All nonfatal errors, strange conditions or status messages during the processing of the routine, should be logged via
\begin{verbatim}
Console, 'what happened' [,/MSG | ,/INFO] 
\end{verbatim}
With this policy the user can decide if he/she actually wants to see all this stuff by changing the console properties.
%
\item Fatal errors should stop the routine at the position of the calling program segment. You may implement a fallback, that the calling function may use to handle the error. Thi should look like:
\begin{verbatim}
PRO RoutineGeneratingError, ERROR=error
On_Error, 2
...
; a fatal error has occured
IF Keyword_Set(ERROR) THEN BEGIN
                             error=1 ; or another documented value/string
                             RETURN
                           END ELSE Console, 'fatal error', /FATAL
...
END
\end{verbatim}
%
\item Console
%
\end{itemize}
\subsection{Syntax Checking}
\begin{itemize}
\item The syntax and semantics of the passed arguments should be checked as strict/extensive as possible. If the routine is time critical (called many times), the checking can be skipped by the keyword \texttt{/FAST}. You also may omit the syntax check, if documented in the \texttt{Restrictions} section. 
\end{itemize}
\subsection{Calling Conventions}

%%%%%%%%%%%%%%%%%%%%%
%%%%%%%%%%%%%%%%%%%%%
\section{Programming}
%%%%%%%%%%%%%%%%%%%%%
%%%%%%%%%%%%%%%%%%%%%


%%%%%%%%%%%%%%%%%%%%%%%%%%%%%%%%%%%%%%%%%%%%
\subsection{Improving Speed and Readability}
%%%%%%%%%%%%%%%%%%%%%%%%%%%%%%%%%%%%%%%%%%%%

\begin{itemize}
\item Use array operations rather than loops wherever possible. Try to avoid loops with high repetition counts.
\item Use IDL system functions and procedures wherever possible.
\item Use \texttt{Call\_Function} or \texttt{Call\_Procedure} instead of \texttt{Execute}.
\item The order in which an expression is evaluated can have a significant effect on program speed. Consider the following statement, where A is an array:
\begin{verbatim}
;Scale A from 0 to 16.
B = A * 16. / MAX(A)
\end{verbatim}
This statement first multiplies every element in A by 16 and then divides each element by the value of the maximum element. The number of operations required is twice the number of elements in A. A much faster way of computing the same result is used in the following statement:
\begin{verbatim}
;Scale A from 0 to 16 using only one array operation.
B = A * (16./MAX(A))
\end{verbatim}
\item Avoid if statements.
\item Whenever possible, vector and array data should always be processed with IDL array operations instead of scalar operations in a loop. 
\item Use constants of the correct type.
\item Eliminate invariant expressions.
\end{itemize}


%%%%%%%%%%%%%%%%%%%%%%%%%%%%%%%%%%%%%%%%
\subsection{Reducing Memory Consumption}
%%%%%%%%%%%%%%%%%%%%%%%%%%%%%%%%%%%%%%%%
\begin{itemize}
\item If virtual memory is a problem, try to tailor your programming to minimize the number of images held in IDL variables. Keep in mind that IDL creates temporary arrays to evaluate expressions involving arrays.  
\item Use the TEMPORARY function: A = TEMPORARY(A) + 1 instead of A = A + 1
\item Avoid memory leaks when dealing with handles. This is easy, if you use the NASE system variable !MH as father for all handles you create. By freeing !MH (for master handle), you can also free the memory for all his children.3
\end{itemize}


%%%%%%%%%%%%%%%%%%%%%%%%%%%
\subsection{Traps and Bugs}
%%%%%%%%%%%%%%%%%%%%%%%%%%%
\begin{itemize}
\item Respect the tag-by-value calling trap.
\end{itemize}

%\bibliography{standards}
%\bibliographystyle{plainnat} 
\end{document}
