%
% FILE:     standards (Master)
%
% LOCATION: /neuro/cartman/user/saam/nase/doc/standards/
%
% AUTHOR:   saam
%
% DATE:     Thu Aug 24 10:29:40 CEST 2000
%
% MODIFICATION HISTORY:
%
%       $Log$
%       Revision 1.9  2000/09/28 07:59:56  saam
%       changes to documentation header subsection
%
%       Revision 1.8  2000/09/26 14:05:27  saam
%       changes to directory structure
%
%       Revision 1.7  2000/09/25 14:13:59  saam
%       updated the doc header
%       appendend short description of console
%
%       Revision 1.6  2000/09/22 18:32:03  saam
%             added some todos
%
%       Revision 1.5  2000/09/21 10:09:17  saam
%             updated doc section
%
%       Revision 1.4  2000/09/09 16:09:47  saam
%            dunno whats changed
%
%       Revision 1.3  2000/08/25 10:09:22  saam
%            compatability added
%
%       Revision 1.2  2000/08/24 16:46:12  saam
%             extended the messaging section
%
%       Revision 1.1  2000/08/24 10:01:32  saam
%             first version
%
%
%-
\documentclass[12pt]{article}


% Page Size & Formatting
\usepackage[a4paper,dvips]{geometry}           % page layout [twoside]
\setlength{\parindent}{0pt}                    % dont indent first paragraph
%\renewcommand{\baselinestretch}{2.0}          % double line spacing
%\usepackage{showlabels}                       % shows labels on side margin
\usepackage{showkeys}                          % prints label, ref, cite and bib keys
\usepackage{draftcopy}                         % displays draft in background of page (ps only!)
\usepackage{changebar}                         % changebars at margin via \cbstart ... \cbend (ps only!)
\usepackage{verbatim}

% References & Citations
\usepackage{lastpage}                          % access last page by \pageref{LastPage}
\usepackage[round]{natbib}                     % citestyle: Author (Year) by \cite or Author, Year with \citet
\usepackage{prettyref}                         % generates references like Figure/Table/Equation x, instead of x 
\newrefformat{fig}{Figure~\ref{#1}}
\usepackage{acronym}                           % make sure acronyms are spelled out at least once

% Language
\usepackage[english]{babel}                    % set the language (german, english, ...)
\usepackage[latin1]{inputenc}                  % allow special characters directly in source

% Headers & Footers
\usepackage{fancyheadings}                     % allows individual header and footers
\pagestyle{fancy}
\lhead{\tiny \$Header$ $\$}                    % include path/name/date/RCS/CVS-version in header
\chead{}
\rhead{}
%\lfoot{}
%\cfoot{}
%\rfoot{}


% Fonts
%\usepackage{times}                            % use times, palatino, ...  ps-fonts instead of cm (dvi won't work!!)
%\usepackage{soul}                             % typeset text in spaced out way via \st{...} and more
\usepackage{SIunits}                           % use SI units by their name and produce upright greek symbols

        
% Figures & Captions
\usepackage{graphicx}
\usepackage[footnotesize,nooneline]{caption}
\DeclareGraphicsRule{.tif}{eps}{.bb}{`convert #1 eps:-}
\DeclareGraphicsRule{.gif}{eps}{.bb}{`convert #1 eps:-}
\DeclareGraphicsRule{.bmp}{eps}{.bb}{`convert #1 eps:-}
\DeclareGraphicsRule{.jpg}{eps}{.bb}{`convert #1 eps:-}
% TeX requirements for putting a figure onto a page
% are rather strict (and therefore seldom fulfilled
% the following command make the placement of figures
% more sloppier
\renewcommand{\textfraction}{0.1}
\renewcommand{\topfraction}{0.9}
\renewcommand{\bottomfraction}{0.9}


% Definitions
% the way a bind figures into the document:
\newcommand{\myfig}[5]{\begin{figure}[tbh]\begin{center}\includegraphics[#3]{#2}\caption{\label{#1}{\sc#4.} #5}\end{center}\end{figure}}


% define an acronym by: 
%   \acrodef{CCH}{cross coincidence histogram} 
% following commands produce the shown output:
%   \ac{CCH}  -> cross coincidence histogram (CCH)  % for the first appearance 
%   \ac{CCH}  -> CCH                                % any further appearance
%   \acf{CCH} -> cross coincidence histogram (CCH)  % always
%   \acl{CCH} -> cross coincidence histogram        % always
%   \acs{CCH} -> CCH                                % always
% you can also make a list of acronyms (see documentation)


% integrate figures by:
%   \myfig{label}{filename}{width,height}{short-title}{description}


\begin{document}
\begin{titlepage}
\begin{center}
{\Huge\textbf{N.A.S.E. / M.I.N.D. Standards}\\[2cm]}
Version: \$Id$ $\$ 
\end{center}
\vfill
\tableofcontents
\vfill
\vfill
\end{titlepage}

%%%%%%%%%%%%%%%%%%%%%%%
%%%%%%%%%%%%%%%%%%%%%%%
\section{Documentation}
%%%%%%%%%%%%%%%%%%%%%%%
%%%%%%%%%%%%%%%%%%%%%%%
\subsection{General}
\begin{itemize}
\item The only accepted documentation language is English. There are lots of old routines documented in German. This has to be changed! If such a routine is modified, it has to be translated. 
\item Documentation is not only for others, it is for you as well. Remember lots of routines you wrote to finish that paper last summer, and now you can only guess what they are doing. And you will probably code them completely new to get them working with a new set of data. Or you forgot, that they exist. This does not only count for the usage, but also for the implementation of your code. So once again: \textbf{Document everything, for you and others.}
\end{itemize}

\subsection{Documentation Header}
A current version of the documentation header can always be found in the doc directory of nase call \texttt{header.pro}.

\verbatiminput{../header.pro}

\begin{itemize}
\item The use of other identifiers as specified above is strictly prohibited.
\item Identifiers mentioned above are needed to syntactically parse the header. Don't use them in your descriptions (e.g., \texttt{See also:}), because this
      confuses scanning and leads to terrible results. Should we restrict tag names to all upper-case, to reduce probability for conflicts?
\item Should we check to correct parsing during commit?
\end{itemize}

\subsection{CVS Logs}
\begin{itemize}
\item Please produce a log message, that other people are able to understand what you have changed. Expressions like \texttt{small bug fix} are not accaptable (what was the bug? how was it fixed?).  
\end{itemize}




%%%%%%%%%%%%%%%%%%
%%%%%%%%%%%%%%%%%%
\section{Routines}
%%%%%%%%%%%%%%%%%%
%%%%%%%%%%%%%%%%%%
\subsection{Naming}
\begin{itemize}
\item Names of routines must not contain any special character like language specific extension, underscores. To keep problems between different operation systems small, just use lower case letters \texttt{a-z}. Names are not restricted in length, but the shorter a name, the better it is to type.  
\item Routine names are expected to describe, what they are supposed to do. Names like \texttt{eval.pro} are not acceptable.
\item Check that the name you like to chose is free for use.
\item If there is a bundle of routines (perhaps handling a special data structure), compound words in increasing specifity should be used. E.g., if you have routines handling a list, they can be called like \texttt{listinsert}, \texttt{listsort}.
\end{itemize}


\subsection{Compatibility between IDL versions}
Different types of compatilibity problems are known between IDL versions. First (the harmless one), higher version include routines, which formerly had to be implemented by hand. Furthermore, already existing routines are extended in their functionality, making several hand-written wrappers obsolete. Last, but most problematic are undocumented (silent) changes in the behaviour of routines, i.e. color management, handling of the \texttt{\_EXTRA} keyword, passing of undefined arguments. The following guidelines are a compromise between functionality and compatability:
\begin{itemize}
\item If a new IDL routines makes a NASE routine obsolete, disable the NASE routine, generate a message about the fact, remove the routine after 2 months from the repository.
\item For changes in IDL's behaviour: all routines must work with (and only with) the current IDL version. The only exception is the NASE/MIND simulation kernel.
\end{itemize}

\subsection{Error Handling and Messaging}
\begin{itemize}
%
\item Wrong number, type, dimensions of positional or keyword parameters should stop the routine at the position of the calling program segment. For this you should place 
\begin{verbatim}
On_Error, 2
\end{verbatim}
(after testing and debugging, of course) at the very beginning of your routine. To produce the actual error message, use
\begin{verbatim}
Console, 'error message', /FATAL .
\end{verbatim}
%
\item All nonfatal errors, strange conditions or status messages during the processing of the routine, should be logged via
\begin{verbatim}
Console, 'what happened' [,/MSG | ,/INFO] 
\end{verbatim}
With this policy the user can decide if he/she actually wants to see all this stuff by changing the console properties.
%
\item Fatal errors should stop the routine at the position of the calling program segment. You may implement a fallback, that the calling function may use to handle the error. This should look like:
\begin{verbatim}
PRO RoutineGeneratingError, ERROR=error
On_Error, 2
...
; a fatal error has occured
IF Keyword_Set(ERROR) THEN BEGIN
                             error=1 ; or another documented value/string
                             RETURN
                           END ELSE Console, 'fatal error', /FATAL
...
END
\end{verbatim}
%
\item The console provides a unified message-mechanism for all
  NASE/MIND routines. The commonly used methods like \texttt{print} and
  \texttt{message} have the disadvantage, that you can't supress these
  logs. Console provides messaging using different importance levels
  (e.g., message, warning, fatal) and the user can decide what level
  to ignore, which to log and when the execution has to be stopped.
\end{itemize}
\subsection{Syntax Checking}
\begin{itemize}
\item The syntax and semantics of the passed arguments should be checked as strict/extensive as possible. If the routine is time critical (called many times), the checking can be skipped by the keyword \texttt{/FAST}. You also may omit the syntax check, if documented in the \texttt{Restrictions} section. 
\end{itemize}
\subsection{Calling Conventions}
\begin{itemize}
\item The time should be the first index in all array operations.
\end{itemize}


%%%%%%%%%%%%%%%%%%%%%
%%%%%%%%%%%%%%%%%%%%%
\section{Programming}
%%%%%%%%%%%%%%%%%%%%%
%%%%%%%%%%%%%%%%%%%%%


%%%%%%%%%%%%%%%%%%%%%%%%%%%%%%%%%%%%%%%%%%%%
\subsection{Improving Speed and Readability}
%%%%%%%%%%%%%%%%%%%%%%%%%%%%%%%%%%%%%%%%%%%%

\begin{itemize}
\item The first index of an array is processed fastest.
\item Use array operations rather than loops wherever possible. Try to avoid loops with high repetition counts.
%
\item Use IDL system functions and procedures wherever possible. A common operation is to find the sum of the elements in an array or subarray. The \texttt{TOTAL} function directly and efficiently evaluates this sum at least 10 times faster than directly coding the sum.
\begin{verbatim}
;Slow way: Initialize SUM and sum each element. 
sum = 0. & FOR I = J, K DO sum = sum + array[I]
;Efficient, simple way.
sum = TOTAL(array[J:K])
\end{verbatim}
%
\item Use \texttt{Call\_Function} or \texttt{Call\_Procedure} instead of \texttt{Execute} wherever possible.
%
\item The order in which an expression is evaluated can have a significant effect on program speed. Consider the following statement, where A is an array:
\begin{verbatim}
;Scale A from 0 to 16.
B = A * 16. / MAX(A)
\end{verbatim}
This statement first multiplies every element in A by 16 and then divides each element by the value of the maximum element. The number of operations required is twice the number of elements in A. A much faster way of computing the same result is used in the following statement:
\begin{verbatim}
;Scale A from 0 to 16 using only one array operation.
B = A * (16./MAX(A))
\end{verbatim}
%
\item Avoid if statements. Programs with array expressions run faster than programs with scalars, loops, and IF statements. 
\begin{verbatim}
;Using a loop will be slow.
FOR I = 0, (N-1) DO IF B[I] GT 0 THEN A[I] = A[I] + B[I]
;Fast way: Mask out negative elements using array operations.
A = A + (B GT 0) * B
;Faster way: Add B > 0.
A = A + (B > 0)
\end{verbatim}

\item Whenever possible, vector and array data should always be processed with IDL array operations instead of scalar operations in a loop. 
%
\item Use constants of the correct type. Consider the following expression:
\begin{verbatim}
A + 5
\end{verbatim}
If the variable \texttt{A} is of floating-point type, the constant \texttt{5} must be converted from short integer type to floating point each time the expression is evaluated. The type of a constant also has an important effect in array expressions. Care must be taken to write constants of the correct type. In particular, when performing arithmetic on byte arrays with the intent of obtaining byte results, be sure to use byte constants; e.g., \texttt{nB}. For example, if A is a byte array, the result of the expression \texttt{A + 5B} is a byte array, while \texttt{A + 5} yields a 16-bit integer array.
%
\item Eliminate invariant expressions. Expressions whose values do not change inside a loop should be moved outside the loop. For example, in the loop:
\begin{verbatim}
FOR I = 0, N - 1 DO arr[I, 2*J-1] = ...,
\end{verbatim}
the expression (2*J-1) is invariant and should be evaluated only once before the loop is entered:
\begin{verbatim}
temp = 2*J-1
FOR I = 0, N-1 DO arr[I, temp] = ....
\end{verbatim}
%
\end{itemize}


%%%%%%%%%%%%%%%%%%%%%%%%%%%%%%%%%%%%%%%%
\subsection{Reducing Memory Consumption}
%%%%%%%%%%%%%%%%%%%%%%%%%%%%%%%%%%%%%%%%
\begin{itemize}
\item If virtual memory is a problem, try to tailor your programming to minimize the number of images held in IDL variables. Keep in mind that IDL creates temporary arrays to evaluate expressions involving arrays.  
\item Use the \texttt{TEMPORARY} function: \texttt{A = TEMPORARY(A) + 1} instead of \texttt{A = A + 1}
\item Avoid memory leaks when dealing with handles. This is easy, if you use the NASE system variable \texttt{!MH} as father for all handles you create. By freeing \texttt{!MH} (for master handle), you can also free the memory for all his children.
\end{itemize}


%%%%%%%%%%%%%%%%%%%%%%%%%%%
\subsection{Traps and Bugs}
%%%%%%%%%%%%%%%%%%%%%%%%%%%
\begin{itemize}
\item Respect the tag-by-value calling trap.
\end{itemize}

%\bibliography{standards}
%\bibliographystyle{plainnat} 
\end{document}
