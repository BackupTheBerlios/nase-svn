%
% FILE: Neurontypes
%
% LOCATION: doc
%
% MODIFICATION HISTORY:
%
%       $Log$
%       Revision 1.5  2004/09/16 14:40:12  thiel
%          Added hint on skipped type '13'.
%
%       Revision 1.4  2004/09/16 14:00:53  thiel
%          Update: added types 12 and 14.
%
%       Revision 1.3  2001/06/12 13:42:10  thiel
%          Added GRNeurons to known types.
%
%       Revision 1.2  2000/10/03 18:15:28  thiel
%           "Contents" instead of "Inhalt".
%
%       Revision 1.1  2000/10/03 17:54:19  thiel
%           neurontype documentation, formerly "neuronen.txt"
%
%-
\documentclass[12pt]{article}


% Page Size & Formatting
\usepackage[a4paper,dvips]{geometry}           % page layout [twoside]
\setlength{\parindent}{0pt}                    % dont indent first paragraph
%\renewcommand{\baselinestretch}{2.0}          % double line spacing
%\usepackage{showlabels}                       % shows labels on side margin
\usepackage{showkeys}                          % prints label, ref, cite and bib keys
\usepackage{draftcopy}                         % displays draft in background of page (ps only!)
\usepackage{changebar}                         % changebars at margin via \cbstart ... \cbend (ps only!)


% References & Citations
\usepackage{lastpage}                          % access last page by \pageref{LastPage}
\usepackage[round]{natbib}                     % citestyle: Author (Year) by \cite or Author, Year with \citet
\usepackage{prettyref}                         % generates references like Figure/Table/Equation x, instead of x 
\newrefformat{fig}{Figure~\ref{#1}}
\usepackage{acronym}                           % make sure acronyms are spelled out at least once

% Version Control (rcs-latex must be installed!)
%\usepackage{rcs}
%\RCS $Header$
%\RCS $keyword$  % for keyword all RCS/CVS tags (like Header, Id) may be inserted; this line defines an tex command
%                % while the following command inserts it into the document
%\RCSkeyword
%                % there is also support for a formatted log history...

% Language
\usepackage[english]{babel}                    % set the language (german, english, ...)
\usepackage[latin1]{inputenc}                  % allow special characters directly in source

% Headers & Footers
\usepackage{fancyheadings}                     % allows individual header and footers
\pagestyle{fancy}
\lhead{}%\tiny\RCSHeader}                        % include path/name/date/RCS/CVS-version in header
\chead{}
\rhead{}
%\lfoot{}
%\cfoot{}
%\rfoot{}


% Fonts
%\usepackage{times}                            % use times, palatino, ...  ps-fonts instead of cm (dvi won't work!!)
%\usepackage{soul}                             % typeset text in spaced out way via \st{...} and more
\usepackage{SIunits}                           % use SI units by their name and produce upright greek symbols

        
% Figures & Captions
\usepackage{graphicx}
\usepackage[footnotesize,nooneline]{caption}
\DeclareGraphicsRule{.tif}{eps}{.bb}{`convert #1 eps:-}
\DeclareGraphicsRule{.gif}{eps}{.bb}{`convert #1 eps:-}
\DeclareGraphicsRule{.bmp}{eps}{.bb}{`convert #1 eps:-}
\DeclareGraphicsRule{.jpg}{eps}{.bb}{`convert #1 eps:-}
% TeX requirements for putting a figure onto a page
% are rather strict (and therefore seldom fulfilled
% the following command make the placement of figures
% more sloppier
\renewcommand{\textfraction}{0.1}
\renewcommand{\topfraction}{0.9}
\renewcommand{\bottomfraction}{0.9}


% Definitions
% the way a bind figures into the document:
\newcommand{\myfig}[5]{\begin{figure}[tbh]\begin{center}\includegraphics[#3]{#2}\caption{\label{#1}{\sc#4.} #5}\end{center}\end{figure}}


% define an acronym by: 
%   \acrodef{CCH}{cross coincidence histogram} 
% following commands produce the shown output:
%   \ac{CCH}  -> cross coincidence histogram (CCH)  % for the first appearance 
%   \ac{CCH}  -> CCH                                % any further appearance
%   \acf{CCH} -> cross coincidence histogram (CCH)  % always
%   \acl{CCH} -> cross coincidence histogram        % always
%   \acs{CCH} -> CCH                                % always
% you can also make a list of acronyms (see documentation)


% integrate figures by:
%   \myfig{label}{filename}{width,height}{short-title}{description}


\begin{document}

\begin{center}
{\Huge\textbf{NASE Neurontypes}}\\[2cm]
Version: \$Id$ $\$ 
\end{center}

\tableofcontents

\section{Graded Response Neuron (GRN)}
\begin{description}
\item[Type:] 'GRN'

\item[Parameters:] transfunc, tauf, taul, sigma, noisystart

\item[Synapses:] feeding, linking, noise

\item[Output:] \texttt{m = f * (1 + l), o = depending on transfer function}

\item[Comments:] Neurons with continuous output function.
\end{description}

\section{Leaky Integrate and Fire Neuron (LIF)}
\begin{description}
\item[Type:] 'LIF'

\item[Parameters:] tauf, taul, taui, th0, sigma, noisystart, spikenoise

\item[Synapses:] feeding, linking, inhibition, noise

\item[Output:] \texttt{m = f * (1 + l) - i, o = m GE th0}

\item[Comments:] First type with real name.
\end{description}



\section{Standard Marburg Model Neuron (SMMN)}
\begin{description}
\item[Type:] '1'

\item[Parameters:] tauf, taul, taui, vs, taus, th0, sigma, noisystart, spikenoise

\item[Synapses:] feeding, linking, inhibition

\item[Output:] \texttt{m = f * (1 + l) - i, o = m GE (s + th0)}

\item[Comments:] LIF +vs +taus
\end{description}



\section{Adaptive Marburg Model Neuron (AMMN)}
\begin{description}
\item[Type:] '2'

\item[Parameters:] tauf, taul, taui, vs, taus, vr, taur, th0, sigma, noisystart,
              spikenoise

\item[Synapses:] feeding, linking, inhibition

\item[Output:] \texttt{m = f * (1 + l) - i, o = m GE (r + s+ th0)}

\item[Comments:] SMMN +vr +taur
\end{description}



\section{Learning Potential Neuron}
\begin{description}
\item[Type:] '3'
\item[Comments:] Removed.
\end{description}



\section{Oversampling Neuron (OSN)}
\begin{description}
\item[Type:] '4'

\item[Parameters:] tauf, taul, taui, vr, taur, vs, taus, th0, sigma, noisystart, 
              oversampling, refperiod, spikenoise, fade

\item[Synapses:] feeding, linking, inhibition

\item[Output:] \texttt{m = f * (1 + l) - i, o = m GE (r + s + th0)}

\item[Comments:] AMMN +oversampling +refperiod +fade
\end{description}



\section{NMDA Linking Neuron}
\begin{description}
\item[Type:] '5'

\item[Parameters:] tauf, taul, taun, taui, vs, taus, th0, sigma

\item[Synapses:] feeding, tp2feeding, linking, nmda, inhibition

\item[Output:] \texttt{m = f * (1 + l + n) - i, o = m GE (s + th0)}

\item[Comments:] SMMN -noisystart -spikenoise +taun, development stopped?
\end{description}



\section{Second Order EPSP Neuron}
\begin{description}
\item[Type:] '6'

\item[Parameters:] [tauf1,tauf2], [taul1,taul2], taui, vr, taur, vs, taus, th0, 
              sigma, noisystart, oversampling, refperiod, spikenoise, fade 

\item[Synapses:] feeding, linking, inhibition, direct

\item[Output:] \texttt{m = (f2-f1) * (1 + (l2-l1)) - i, o = m GE (r + s + th0)}

\item[Comments:] OSN -tauf -taul +[tauf1,tauf2] +[taul1,taul2]
\end{description}



\section{Two Feeding Inhibition Neuron}
\begin{description}
\item[Type:] '7'

\item[Parameters:] tauf1, tauf2, taul, taui1, taui2, vr, taur, vs, taus, th0, 
              sigma, noisystart, spikenoise

\item[Synapses:] feeding1, feeding2, linking, inhibition1, inhibition2

\item[Output:] \texttt{m = (f1 + f2 - i1) * (1 + l) - i2, o = m GE (r+ s + th0)}

\item[Comments:] AMMN -tauf -taui +tauf1 +tauf2 +taui1 +taui2
\end{description}



\section{Four Compartment Neuron (FCN)}
\begin{description}
\item[Type:] '8'

\item[Parameters:] deltat, th, refperiod, tausyn, vna, vk, vl, vsyn, gna, gk, gm, 
                gc, gsyn, cm, soma\_d, soma\_l, den1\_d, den1\_l, den2\_d, den2\_l,
                den3\_d, den3\_l

\item[Synapses:] curr1, curr2, curr3, syn1, syn2, syn3

\item[Output:] \texttt{v = Hodgkin-Huxley fast-slow equations, o = v GE th0}

\item[Comments:] Hodgkin-Huxley.
\end{description}



\section{Four Compartment Object Neuron}
\begin{description}
\item[Type:] '9'

\item[Parameters:] deltat, th, refperiod, tausyn, vna, vk, vl, vsyn, gna, gk, gm,
              gc, gsyn, cm, soma\_d, soma\_l, den1\_d, den1\_l, den2\_d, den2\_l,
              den3\_d, den3\_l

\item[Synapses:] curr1, curr2, curr3, syn1, syn2, syn3

\item[Output:] \texttt{v = Hodgkin-Huxley fast-slow equations, o = v GE th0}

\item[Comments:] FCN
\end{description}



\section{Poisson Neuron}
\begin{description}
\item[Type:] '10'

\item[Parameters:] tauf, probability

\item[Synapses:] feeding

\item[Output:] \texttt{r = RandomU, o = r LE (probability + f)}

\item[Comments:] probability is defined in \texttt{InitLayer\_10}.
\end{description}



\section{Inhibitory Linking Neuron}
\begin{description}
\item[Type:] '11'

\item[Parameters:] tauf, taul1, taul2, taui, vs, taus, th0, sigma, noisystart, 
              spikenoise

\item[Synapses:] feeding, linking, ilinking, inhibition

\item[Output:] \texttt{m = f * (1 + ((l1 - l2) > 0)) - i, o = m GE (s + th0)}

\item[Comments:] SMMN -taul +taul1 +taul2 
\end{description}



\section{Shunting Marburg Model Neuron}
\begin{description}
\item[Type:] '12'

\item[Parameters:] tauf, taul, taui, taux, vs, taus, th0, sigma, noisystart, 
              spikenoise

\item[Synapses:] feeding, linking, inhibition, shunting

\item[Output:] \texttt{f = f * exp(-1/tauf * sqrt((1-x) > 0.0)),\\
 m = f * (1 + l) - i, o = m GE (s + th0)}

\item[Comments:] SMMN +taux 
\end{description}



\section{Two Dendrite Neuron}
\begin{description}
\item[Type:] '14'

\item[Parameters:] tauf1, tauf2, taul, taui1, taui2, vs, taus, vr,
                   taur, th0, sigma, noisystart, spikenoise

\item[Synapses:] feeding1, feeding2, linking, inhibition1, inhibition2

\item[Output:] \texttt{m = (f2 + (1+l) * (f1-i1) / (1+i2)), o = m GE (r+ s + th0)}

\item[Comments:] Type '13' seems to have been skipped.
\end{description}
\end{document}
