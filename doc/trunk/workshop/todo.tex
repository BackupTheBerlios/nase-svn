%
% FILE:     workshop (Master)
%
% LOCATION: /mhome/saam/sim/doc/workshop/
%
% AUTHOR:   saam
%
% DATE:     Sun Oct  1 18:59:58 MEST 2000
%
% MODIFICATION HISTORY:
%
%       $Log$
%       Revision 1.3  2000/10/03 16:16:51  saam
%       updated
%
%       Revision 1.2  2000/10/02 16:39:52  saam
%       todos from monday inserted
%
%       Revision 1.1  2000/10/02 16:35:10  saam
%       append one entry
%
%
%-
\documentclass[12pt]{article}


% Page Size & Formatting
\usepackage[a4paper,dvips]{geometry}           % page layout [twoside]
\setlength{\parindent}{0pt}                    % dont indent first paragraph
%\renewcommand{\baselinestretch}{2.0}          % double line spacing
%\usepackage{showlabels}                       % shows labels on side margin
\usepackage{showkeys}                          % prints label, ref, cite and bib keys
\usepackage{draftcopy}                         % displays draft in background of page (ps only!)
\usepackage{changebar}                         % changebars at margin via \cbstart ... \cbend (ps only!)


% References & Citations
\usepackage{lastpage}                          % access last page by \pageref{LastPage}
\usepackage[round]{natbib}                     % citestyle: Author (Year) by \cite or Author, Year with \citet
\usepackage{prettyref}                         % generates references like Figure/Table/Equation x, instead of x 
\newrefformat{fig}{Figure~\ref{#1}}
\usepackage{acronym}                           % make sure acronyms are spelled out at least once

% Version Control (rcs-latex must be installed!)
\usepackage{rcs}
\RCS $Header$
%\RCS $keyword$  % for keyword all RCS/CVS tags (like Header, Id) may be inserted; this line defines an tex command
%                % while the following command inserts it into the document
%\RCSkeyword
%                % there is also support for a formatted log history...

% Language
\usepackage[english,german]{babel}                    % set the language (german, english, ...)
\usepackage[latin1]{inputenc}                  % allow special characters directly in source

% Headers & Footers
\usepackage{fancyheadings}                     % allows individual header and footers
\pagestyle{fancy}
\lhead{\tiny\RCSHeader}                        % include path/name/date/RCS/CVS-version in header
\chead{}
\rhead{}
%\lfoot{}
%\cfoot{}
%\rfoot{}


% Fonts
%\usepackage{times}                            % use times, palatino, ...  ps-fonts instead of cm (dvi won't work!!)
%\usepackage{soul}                             % typeset text in spaced out way via \st{...} and more
\usepackage{SIunits}                           % use SI units by their name and produce upright greek symbols

        
% Figures & Captions
\usepackage{graphicx}
\usepackage[footnotesize,nooneline]{caption}
\DeclareGraphicsRule{.tif}{eps}{.bb}{`convert #1 eps:-}
\DeclareGraphicsRule{.gif}{eps}{.bb}{`convert #1 eps:-}
\DeclareGraphicsRule{.bmp}{eps}{.bb}{`convert #1 eps:-}
\DeclareGraphicsRule{.jpg}{eps}{.bb}{`convert #1 eps:-}
% TeX requirements for putting a figure onto a page
% are rather strict (and therefore seldom fulfilled
% the following command make the placement of figures
% more sloppier
\renewcommand{\textfraction}{0.1}
\renewcommand{\topfraction}{0.9}
\renewcommand{\bottomfraction}{0.9}


% Definitions
% the way a bind figures into the document:
\newcommand{\myfig}[5]{\begin{figure}[tbh]\begin{center}\includegraphics[#3]{#2}\caption{\label{#1}{\sc#4.} #5}\end{center}\end{figure}}


% define an acronym by: 
%   \acrodef{CCH}{cross coincidence histogram} 
% following commands produce the shown output:
%   \ac{CCH}  -> cross coincidence histogram (CCH)  % for the first appearance 
%   \ac{CCH}  -> CCH                                % any further appearance
%   \acf{CCH} -> cross coincidence histogram (CCH)  % always
%   \acl{CCH} -> cross coincidence histogram        % always
%   \acs{CCH} -> CCH                                % always
% you can also make a list of acronyms (see documentation)


% integrate figures by:
%   \myfig{label}{filename}{width,height}{short-title}{description}


\begin{document}

\begin{center}
{\Huge\textbf{Bugs, ToDos and Feature Reqests}}\\[2cm]
Version: \$Id$ $\$ 
\end{center}

\section{Help System \& Documentation Header} 

\begin{itemize}
\item subscribe Andreas Bruns to mailing list
\item Category: Kombination aus kurzen Keywords (feste Liste) zum Suchen, Syntax-Check?
\item \texttt{Keyword Parameters} heissen jetzt \texttt{Input Keywords}
\item Funktionen werden durch ein nachgestelltes rundes Klammerpaar angedeutet.
\item Keywords werden komplett grossgeschrieben. 
\item Abtrennung von Keywords und Beschreibung durch ``::''
\item show path to file in header display
\item standards Makefile broken for non-saam environment
\end{itemize}



\section{Control}
\begin{itemize}
\item allow \texttt{console} to log into a file, simultaneously
\item console ignores level keyword and outputs filename instead of procedure/function name
\item \texttt{SimTime}: make it work together with console
\end{itemize}


\section{Graphic}
\begin{itemize}
\item \texttt{PlotTvScl}: PS and NASE option broken
\item \texttt{PlotTvScl}: colors in PS broken
\item \texttt{CAW}: doesn't work for Andreas
\end{itemize}


\section{Methods}
\begin{itemize}
\item \texttt{peakPicker}: strange or wrong keyword handling
\end{itemize}


\section{MIND}

\begin{itemize}
\item routines save their evaluation parameters 
\item allow deletion of unnecessary tags
\item external learning rules should display relevant information
\item \texttt{DSim}: not up to date
\item \texttt{FakeEach}: omit the skel keyword, or delete from documentation
\item modify \texttt{foreach} to process multiple parameter variations synchronously. 
\end{itemize}



\section{Misc}
\begin{itemize}
\item \texttt{StrJoin}: internal IDL routine?
\item How can marburg specific extensions be included?
\item \texttt{RealFileName}: marburg specific actions brake other systems
\end{itemize}




\section{Simulation Kernel}

\begin{itemize}
\item improve \texttt{mSetSDW} with \texttt{temporary}
\item \texttt{saveDW} and \texttt{restoreDW} should work with \texttt{SDW}
\item initialize T2C on demand and in a separate function
\item extend \texttt{initDW} to work \texttt{SLOW} and \texttt{FAST} version
\item check the functionality of weight display routines, when T2C is missing
\item \texttt{delayWeigh}: improvement in processing speed and memory consumption?
\item \texttt{floatWeigh}: commit, leave a message in \texttt{delayWeigh} to perform all changes in \texttt{floatWeigh} as well
\item commit neuron type file
\end{itemize}



\section{Fulfilled ToDos}



\end{document}
