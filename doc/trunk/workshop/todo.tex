% FILE:     workshop (Master)
%
% LOCATION: /mhome/saam/sim/doc/workshop/
%
% AUTHOR:   saam
%
% DATE:     Sun Oct  1 18:59:58 MEST 2000
%-
\documentclass[12pt]{article}


% Page Size & Formatting
\usepackage[a4paper,dvips]{geometry}           % page layout [twoside]
\setlength{\parindent}{0pt}                    % dont indent first paragraph
%\renewcommand{\baselinestretch}{2.0}          % double line spacing
%\usepackage{showlabels}                       % shows labels on side margin
\usepackage{showkeys}                          % prints label, ref, cite and bib keys
\usepackage{draftcopy}                         % displays draft in background of page (ps only!)
\usepackage{changebar}                         % changebars at margin via \cbstart ... \cbend (ps only!)


% References & Citations
\usepackage{lastpage}                          % access last page by \pageref{LastPage}
\usepackage[round]{natbib}                     % citestyle: Author (Year) by \cite or Author, Year with \citet
\usepackage{prettyref}                         % generates references like Figure/Table/Equation x, instead of x 
\newrefformat{fig}{Figure~\ref{#1}}
\usepackage{acronym}                           % make sure acronyms are spelled out at least once

% Version Control (rcs-latex must be installed!)
\usepackage{rcs}
\RCS $Header$
%\RCS $keyword$  % for keyword all RCS/CVS tags (like Header, Id) may be inserted; this line defines an tex command
%                % while the following command inserts it into the document
%\RCSkeyword
%                % there is also support for a formatted log history...

% Language
\usepackage[english]{babel}                    % set the language (german, english, ...)
\usepackage[latin1]{inputenc}                  % allow special characters directly in source

% Headers & Footers
\usepackage{fancyheadings}                     % allows individual header and footers
\pagestyle{fancy}
\lhead{\tiny\RCSHeader}                        % include path/name/date/RCS/CVS-version in header
\chead{}
\rhead{}
%\lfoot{}
%\cfoot{}
%\rfoot{}


% Fonts
%\usepackage{times}                            % use times, palatino, ...  ps-fonts instead of cm (dvi won't work!!)
%\usepackage{soul}                             % typeset text in spaced out way via \st{...} and more
\usepackage{SIunits}                           % use SI units by their name and produce upright greek symbols

        
\begin{document}

\begin{center}
{\Huge\textbf{Bugs, ToDos and Feature Requests}}\\[2cm]
Version: \$Id$ $\$ 
\end{center}


\section{How to use this file}

The to-do-list is maintained by the NASE developers. The HTML-page you
are viewing is automatically generated from a LaTeX file, which is
under CVS control. Each developer may check out, modify, and check in
this list. To retrieve a local copy, use "\texttt{cvs checkout todo}".
This will get you a copy of the LaTeX file, as well as the
\texttt{bugs} subdirectory from the normal NASE tree.
    
If you discover a bug in a NASE routine or any issue that should be
looked after, please try to go and have it fixed right away. If this
is not possible at that moment, put a short description of the issue
into this list. You may create new sections as appropriate. If you
discovered a bug, please create a short, commented program that
reproduces it, and put it into the \texttt{bugs} directory.

If you are bored, refer to this list and resolve as many issues as
possible \texttt{;-)}. As soon as an issue is resolved, move it from
its current position in the list to the "Fulfilled ToDos" section
below.

Please note that most LaTeX commands and environments are either not supported by
htmllatex or our stylesheet and cannot be used in this file (this is
listed as a bug to resolve below.)

Thank you!\\
R�diger\\[2cm]



\section{Help System \& Documentation Header} 
\begin{itemize}
\item most LaTeX commands and environments are either not supported by
  htmllatex or our stylesheet.
\item standards Makefile broken for non-saam environment
\item move Traps and Bugs into the standards document
\item \texttt{WhoUses} doesn't work and its wrapper calls routine with wrong path.
\item Parser confuses occasional \texttt{NAME::}--keywords with name
  of the routine given by the \texttt{NAME:} statement in the header.
\item nasecvscommit.ps-File seems to be broken (see my mail to the
  NASE-list from aug-15-2003): In the "nase" CVS module, syntax check
  seems to pass always. In the "CVSROOT" CVS module, the script fails,
  causing CVS to cancel the commit process. This prevents checking in
  of any CVS administration files!
  For a workaround to this problem, see the mail.
\item Adjust leftover ``old'' headers to NASE standard, and translate
  them into Enlish.
\end{itemize}

 
\section{Control}
\begin{itemize}
\item \texttt{console} fails, if xwindow id closed and a console command occurs
\item \texttt{console} should not be the son of !MH
\item \texttt{Video}: extension to save and restore arbitrary parameter structures
\end{itemize}


\section{Graphic}
\begin{itemize}
  \item \texttt{UTvScl}:
    \begin{itemize}
      \item Handling of 1-dim arrays is broken.
      \item Handling of arrays with trailing dimensions of 1 is broken.
      \item When /POLY is set, and only one positioning-argument is given,
        it is interpreted as the x-coordinate, and y defaults to zero.
        This is wrong, positions in the window should be counted in this
        case, see documentation of IDL's TV or POLY=0-behaviour.
      \item Now that UTvScl does handling of !NONEs, it would really by
        nice, if they could be preserved when using CUBIC or INTERP.
        Currently, interpolation breaks handling of !NONEs. Idea: resize
        separate array of nones, and imprint it on the result. Or: Use
        value that stays unchaned during interpolation, e.g. !NaN. Would
        be elegant, but then !NaN must not be contained for other reasons!
    \end{itemize}
  \item \texttt{Examineit}:
    \begin{itemize}
      \item When /NASE is not set,``bound'' display type fails when array contains !NONE
        (value -999999). This originally was a feature, not a bug. But
        as now UTvScl displays !NONE in blue in any case, it should
        always be treated as missing.
      \item breaks, if /MODAL is set (idl help says, cannot have a
        menu bar on a modal top level base -- did this change
        recently?)
    \end{itemize}
  \item NaseTv and NaseTvScl are depricated and should be removed.
  \item \texttt{widget\_image\_container}:
    \begin{itemize}
      \item When renew\_scaling-method is called before widget is
        realized, all options that were passed during intialization
        will be forgotten. (renew\_scaling overwrites (*self).extra.
        Should better only add.)
      \item Interactive selection of palette (inherited from
        basic\_draw\_widget) does break when /LEGEND is set.
    \end{itemize}
  \item \texttt{basic_draw_widget}:
    \begin{itemize}
      \item In interactive palette selection, when ``function'' is
        selected, the next paint event plots into XLOADCT's window.
    \end{itemize}
  \item \texttt{SHOWWEIGHTS_SCALE}: Breaks if called inside the FaceIt
    widget and says:\\
    (30)SHOWWEIGHTS_SCALE: FAILED ASSERTION: 'Range gt 0' in line
    169.\\
    (30)SHOWWEIGHTS_SCALE: Reason: Range_In must be a positive scalar
    value for this routine.\\
    Is this a general bug or only inside Faceit? Find out where the
    bug is and kill it..
\end{itemize}


\section{Methods}
\begin{itemize}
\item Nothing to do here at the moment.
\end{itemize}


\section{MIND}
\begin{itemize}
\item simulation and analysis routines save their evaluation parameters via \texttt{Video}s
\item allow deletion of unnecessary tags
\item external learning rules should display relevant information
\item display iteration time in \texttt{console} instead of terminal
\item Plotting input with /NASE option should be possible
\item Only plot changing inputs or use \texttt{PlotTvScl_Update}.
\end{itemize}



\section{Misc}
\begin{itemize}
\item \texttt{StrJoin}: remove or maintain due to IDL 3.6?
\item \texttt{Split}: bug when working with separator longer than 1 character
\item How can marburg specific extensions be included?
\item \texttt{XMatEdit}: obsolete 
\item \texttt{UOpenR} and partners should use IDLs streaming ZIP functionality.
WARNING: IDL is only able to zip since V5.3!
\end{itemize}



\section{Compatibility Issues}
The following issues exist for backwards compatibility. The ToDos
below should be fulfilled as soon as we decide that backwards
compatibility need not be maintained in this case.
\begin{itemize}
  \item !NONE is defined as -999999. As this value could appear by
    chance in an array, it should be replaced by some special value,
    for example NaN. To make things worse, in \texttt{ExamineIt}, the
    value -999999 is replaced by +999999, in order to work around a
    deficit of IDL~3. (Anyone still using IDL~3, or can we finally drop
    this??)
\end{itemize}




\section{Simulation Kernel}
\begin{itemize}
\item \texttt{saveDW} and \texttt{restoreDW} should work with \texttt{SDW}
\item initialize T2C on demand and in a separate function
\item extend \texttt{initDW} to work \texttt{SLOW} and \texttt{FAST} version
\item check the functionality of weight display routines, when T2C is missing
\item \texttt{N_Spikes} and \texttt{LayerMUA} are redundant. Current status: \texttt{N_Spikes} uses \texttt{LayerMUA} and generates warning.
\item \texttt{SetWeights} new version doesnt delete \texttt{!NONE} connections properly.
\end{itemize}



\section{Fulfilled ToDos}
\label{sec:fulfilled}
\begin{itemize}
\item LExtrac: removed because obsolete [done, AT]
\item \texttt{DelayWeigh}: improvement in processing speed and memory consumption? NOPE [done]
\item Improve \texttt{MSetSDW} with \texttt{temporary}: no improvement [done, BAS]
\item \texttt{Console} ignores level keyword and outputs filename instead of procedure/function name [fixed, BAS]
\item \texttt{CAW}: doesn't work for Andreas, but \texttt{Widget_Control, /RESET} does.
\item \texttt{PeakPicker}: strange or wrong keyword handling. [fixed, AT]
\item Modify \texttt{foreach} to process multiple parameter variations synchronously. [done, MS & AT] 
\item Commit neuron type file [done, AT]
\item Subscribe Andreas Bruns to mailing list [done, AT] 
\item \texttt{DSim}: not up to date [done, AT]
\item Help System: \texttt{Keyword Parameters} heissen jetzt \texttt{Input Keywords} [MS]
\item Help System: Funktionen werden durch ein nachgestelltes rundes Klammerpaar angedeutet [MS]
\item Help System: Keywords werden komplett grossgeschrieben [MS]
\item Help System: Abtrennung von Keywords und Beschreibung durch ``::'' [MS]
\item Help System: show path to file in header display [MS]
\item Help System: header view also with currently edited [MS]
\item \texttt{FakeEach}: omit the skel keyword, or delete from documentation [done, AT]
\item let user parse arbitrary headers before checkin [done, MS]
\item Category: Kombination aus kurzen Keywords (feste Liste) zum Suchen, Syntax-Check? [done, MS]
\item directories unsorted in HyperHelp [done, MS]
\item remove headerdoc.pro and create link on mainpage to doc/header.pro [done, MS]
\item title for doc/header.pro is distorted [done, MS] 
\item allow \texttt{console} to log into a file, simultaneously [done, MS]
\item \texttt{PlotTvScl}: PS and NASE option broken [i think ruediger fixed this, MS] 
\item \texttt{PlotTvScl}: legend only approximately identical with actual coordinate system [see new routine PTVS, done, MS]
\item \texttt{SimTime}: make it work together with console [done, AT] 
\item \texttt{RealFileName}: marburg specific actions brake other systems [done, AT]
\item \texttt{floatWeigh}: commit, leave a message in \texttt{delayWeigh} to perform all changes in \texttt{floatWeigh} as well [FloatWeigh function is now included in DelayWeigh, AT].
\item The To Do--list on the WWW page needs to be updated by
  hand. [done, now updated automatically, RK]
\item The /NASE keyword has inconsistent semantics: sometimes it
  refers to color scaling, sometimes it refers to transposition
  only. [implemented semantically clear replacement keywords NORDER
  and NSCALE, done, RK]
\item 
  In Windows, ExamineIt gives an error message 
  \texttt{WIDGET_CONTROL: Microsoft Windows Error: font name too long.}
  [font definition is now done OS specific in DefGlobVars, AT] 
\item 
  Modify widgets to consisently use the new global font
  \texttt{!NASEWIDGETFONT}.[Apart from ExamineIt, only FaceIt used
  MyFont, which has been changed, AT.] 
\end{itemize}


\end{document}
